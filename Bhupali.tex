
\documentclass{hindustani}

\begin{document}

\title{Raag Bhupali}
\author{}
\date{}
\maketitle

\raag{Bhupali}

Raag Bhupali also called Bhup evokes an emotion of \textit{bhakti} and \textit{shanti}. The notes {\ma}  and {\Nii}  are completely skipped while playing this raag. Bhupali is a \textbf{Audav - Audav} jaati (Pentatonic scale) raag.
The equivalent Raag in Carntic music is \textbf{mohanam}

\aaroha{\cn{sRGPDs2}}

\avaroha{\cn{s2DPGRs}}

Thaat: \textbf{Kalyan}

Time: \textbf{7pm to 10pm}

Vadi Swar: \textbf{\Ga}  

Samvadi swar: \textbf{\Dha}

The Pakad of the raag is \textbf{\cn{GRsD1} \cn{sRG} \cn{PGDPGRs}}


\section{Alankaar}
\begin{itemize}

\item \alankar{\cn{sRG RGP GPD PDs2}}{\cn{s2DP DPG PGR GRs}}
\item \alankar{\cn{sRGP RGPD GPDs2}}{\cn{s2DPG DPGR PGRs}}
\item \alankar{\cn{sRGPD RGPDS2}}{\cn{s2DPGR DPGRs}}
\item \alankar{\cn{sG RP GD Ps2}}{\cn{s2P DG PR Gs}}
\item \alankar{\cn{sGR RPG GDP Ps2D}}{\cn{s2PD DGP PRG GsR}}

\end{itemize}


\section{Composition in Teentaal}
\begin{flushleft}

\teentaal{-,-,-,-,-,-,-,-,D,s,D,P,G,R,s,R,
G,-,G,P,G,R,s,-,s,R,G,P,R,G,P,D,
G,P,D,P,G,R,S,-,G,G,G,G,P,-,D,P,
s2,-,s2,s2,D,R,s2,-,
g2,g2,R2,s2, R2,R2,s2,D,s2,-,D,P,G,R,s2,-}

\teentaal{-,-,-,-,-,-,-,-,D,s,D,P,G,R,s,R,
s,-,-,-,-,-,-,-,D,s,D,P,G,R,s,R,
D,-,-,-,-,-,-,-,D,s,D,P,G,R,s,R,
s,-,-,-,D,-,-,-,D,s,D,P,G,R,s,R,
s,-,D,P,S,D,S,-,D,s,D,P,G,R,s,R,
PDsR*,G,-,R,-,S,,-,-}
\end{flushleft}

\subsection{Taan}
\begin{flushleft}
\teentaal{-,-,-,-,-,-,-,-,D,s,D,P,G,R,s,R,sRG-*, RGP-*,GPDP*, GRs-*}
\teentaal{-,-,-,-,-,-,-,-,D,s,D,P,G,R,s,R,sRG*, sRGP*,sRGPD*, PGRs*}
\teentaal{-,-,-,-,-,-,-,-,D,s,D,P,G,R,s,R,sD1D1s*, sD1D1s*,sD1s-*, sD1Rs*}
\teentaal{-,-,-,-,-,-,-,-,D,s,D,P,G,R,s,R,GRs-*, GRs-*,GPD-*, GRs-*}

\end{flushleft}

\section{Composition in RoopakTaal}
\begin{flushleft}

\roopaktaal{-,-,-,-,-,DP*, GR*,
G-*,-,RG*,PD*, s, sD*,sR*,
GR*,G-*, RG*,PD*,s2}
\end{flushleft}

\end{document}